\chapter{Giriş}

\section{İleri Seviye ROS2 ve Python Kullanımının Önemi}
\subsection{İleri Seviye ROS2 Kullanımının Faydaları}
İleri seviye ROS2 kullanımı, robotik uygulamalar geliştirmede önemli faydalar sağlar. İleri düzey bilgi ve becerilere sahip olmak, daha etkin ve karmaşık robot sistemleri geliştirmenize olanak tanır. Ayrıca, performans ve optimizasyonu artırarak daha hızlı ve verimli sistemler oluşturabilirsiniz. İleri düzey ROS2 kullanımının faydalarından bazıları şunlardır:

\subsubsection{Etkin Robot Sistemleri}
İleri seviye ROS2 kullanımı sayesinde, daha etkin ve karmaşık robot sistemleri geliştirilebilir. Bu, daha gelişmiş robot kontrol ve planlama algoritmaları, insan-robot etkileşimi ve yapay zeka entegrasyonu gibi alanlarda daha başarılı projeler gerçekleştirmenize yardımcı olacaktır.

\subsubsection{Yüksek Performans ve Optimizasyon}
İleri seviye ROS2 teknikleriyle, performans ve optimizasyonu artırarak daha hızlı ve verimli sistemler oluşturabilirsiniz. Bu, daha düşük gecikme süreleri, daha iyi enerji verimliliği ve daha yüksek güvenilirlik gibi önemli avantajlar sağlar.

\subsection{Python ile Etkin Geliştirme}
Python, etkin ve hızlı bir şekilde robotik uygulamalar geliştirmenize olanak tanır. Python'un sunduğu yapılar sayesinde, hızlı ve esnek kod geliştirme süreçleri sağlanabilir. Ayrıca, Python'un geniş kütüphane desteği, ROS2 projelerinde de kullanılabilir ve bu sayede daha kapsamlı işlemler gerçekleştirilebilir. Python ile etkin geliştirme konusunda şu faydaları göz önünde bulundurabilirsiniz:

\subsubsection{Hızlı Kod Geliştirme}
Python dilinin sunduğu yapılar sayesinde, hızlı ve esnek kod geliştirme süreçleri sağlanabilir. Bu, prototip oluşturma aşamasından ürünleşmeye kadar her aşamada zaman kazandırır ve projenizin daha hızlı ilerlemesine yardımcı olur.

\subsubsection{Büyük Kütüphane Desteği}
Python dilinin geniş kütüphane desteği, ROS2 projelerinde de kullanılabilir ve bu sayede daha kapsamlı işlemler gerçekleştirilebilir. Örneğin, Python'da mevcut olan veri analizi, görüntü işleme ve yapay zeka kütüphaneleri sayesinde, robotik uygulamalarınızın yeteneklerini önemli ölçüde artırabilirsiniz.
\section{İçeriğin Amacı ve Kapsamı}
\subsection{İçeriğin Amacı}
Bu içeriğin amacı, temel ROS2 bilgisine sahip kullanıcıların, ileri düzey ROS2 ve Python kullanımı konusunda bilgi ve beceri sahibi olmalarına yardımcı olmaktır. Bu sayede kullanıcılar, daha karmaşık ve etkin robot sistemleri geliştirebilir, performans ve güvenilirlik açısından daha iyi sonuçlar elde edebilirler.

\subsection{İçeriğin Kapsamı}
Bu içerik, aşağıdaki konuları kapsamaktadır:

\begin{enumerate}
\item ROS2 bileşenleri ve mimarisi
\item İleri düzey paket oluşturma ve yönetimi
\item ROS2 güvenlik (SROS2)
\item ROS2 ve Python ile test ve entegrasyon
\item ROS2 ve gerçek zamanlı sistemler (Real-Time)
\item İleri düzey TF2 kullanımı ve optimizasyon
\item ROS2 ile simülasyon ve donanım entegrasyonu
\item İleri düzey araçlar ve kullanımları
\item ROS2 ve ileri düzey uygulamalar ve örnek projeler
\end{enumerate}

İçeriğin kapsamı boyunca, bu konularla ilgili temel kavramlar, yöntemler ve pratik uygulamalar ele alınacaktır.