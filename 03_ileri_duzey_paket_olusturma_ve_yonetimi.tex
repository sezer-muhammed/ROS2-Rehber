% 03_ileri_duzey_paket_olusturma_ve_yonetimi.tex
\chapter{İleri Düzey Paket Oluşturma ve Yönetimi}

\section{Paket Yapılandırmasında Pluginlib ve Plugin Sistemleri}
\subsection{Pluginlib Kütüphanesi ve Kullanımı}
% Pluginlib hakkında kısa bilgi
\lipsum[1] % Örnek metin için, gerçek içerikle değiştirin

\subsection{Plugin Sistemleri ve Paket Yapılandırması}
% Plugin sistemlerinin avantajları ve kullanım alanları
\lipsum[2] % Örnek metin için, gerçek içerikle değiştirin

\subsubsection{Plugin Tanımlama ve Kullanma}
% Pluginlerin nasıl tanımlandığı ve kullanıldığı hakkında bilgi
\lipsum[3] % Örnek metin için, gerçek içerikle değiştirin

\subsubsection{Plugin Yönetimi ve Entegrasyonu}
% Plugin yönetimi ve entegrasyonu ile ilgili bilgi
\lipsum[4] % Örnek metin için, gerçek içerikle değiştirin

\section{Özel Mesaj, Hizmet ve Eylem Türleri Oluşturma}
\subsection{Özel Mesaj Türleri}
% Özel mesaj türlerinin tanımlanması ve kullanılması hakkında bilgi
\lipsum[5] % Örnek metin için, gerçek içerikle değiştirin

\subsection{Özel Hizmet Türleri}
% Özel hizmet türlerinin tanımlanması ve kullanılması hakkında bilgi
\lipsum[6] % Örnek metin için, gerçek içerikle değiştirin

\subsection{Özel Eylem Türleri}
% Özel eylem türlerinin tanımlanması ve kullanılması hakkında bilgi
\lipsum[7] % Örnek metin için, gerçek içerikle değiştirin
