\chapter{İleri Düzey Paket Oluşturma ve Yönetimi}

\section{Paket Oluşturma, Yapılandırma ve Yönetimi}
    \subsection{ROS2 Paketi Oluşturma ve Yapılandırma}
        \subsubsection{Paket Oluşturma ve Gerekli Dosyalar}
            ROS2 paketlerini oluşturmak için öncelikle `ros2 pkg create` komutunu kullanırız. Bu komut ile paketimize gerekli dosyalar ve klasörler otomatik olarak oluşturulur. Bu dosyalar arasında `CMakeLists.txt`, `package.xml`, `src` ve `include` gibi önemli yapılandırma ve kaynak dosyaları bulunmaktadır.

        \subsubsection{CMakeLists.txt ve package.xml Dosyaları}
            `CMakeLists.txt` dosyası, paketimizin derleme ve bağlantı işlemleri için gerekli yapılandırmaları içerir. Bu dosyada bağımlılıklar, hedefler ve derleme seçenekleri gibi bilgiler yer alır.

            `package.xml` dosyası ise, paketimizin meta bilgilerini ve bağımlılıklarını içerir. Bu dosyada paket adı, sürümü, lisansı ve yazarları gibi bilgilerin yanı sıra, diğer ROS2 paketleri ile olan bağımlılıklar da belirtilir.

    \subsection{Paket İçindeki Launch Dosyaları ve Kullanımı}
        \subsubsection{Launch Dosyalarının Oluşturulması ve Yapısı}
            Launch dosyaları, ROS2 düğümlerini başlatmak ve yönetmek için kullanılır. Launch dosyaları `.launch.py` uzantısıyla oluşturulur ve Python programlama dili kullanılarak yazılır. Bu dosyalar, düğüm yapılandırmalarını ve parametrelerini içerir ve genellikle `launch` klasörü içinde bulunur.

        \subsubsection{Paket İçindeki Birden Fazla Node'u Başlatma}
            Launch dosyaları sayesinde, bir paket içinde birden fazla düğümü aynı anda başlatıp yönetebiliriz. Bu işlem, launch dosyasında `Node` sınıfı kullanılarak gerçekleştirilir ve düğümler arasındaki bağımlılıklar ve iletişim ayarları yapılandırılır.

    \subsection{ROS2 Paketleri Arasında Bağımlılıklar ve İletişim}
        \subsubsection{Paket Bağımlılıklarının Yönetimi}
            Paketler arasındaki bağımlılıklar, `package.xml` ve `CMakeLists.txt` dosyalarında yönetilir. Paketler arasındaki bağımlılıklar eklenerek, hizmetler ve iletişim kanalları aracılığıyla veri alışverişi sağlanır.


\subsubsection{İletişim ve Hizmetlerle Paketler Arasında Veri Alışverişi}
% Hizmetler ve iletişim kanalları aracılığıyla paketler arasında veri alışverişi

\section{Özel Mesaj, Hizmet ve Eylem Türleri Oluşturma}
\subsection{Özel Mesaj Türleri}
% Özel mesaj türlerinin tanımlanması ve kullanımı hakkında bilgi

\subsection{Özel Hizmet Türleri}
% Özel hizmet türlerinin tanımlanması ve kullanımı hakkında bilgi

\subsection{Özel Eylem Türleri}
% Özel eylem türlerinin tanımlanması ve kullanımı hakkında bilgi



