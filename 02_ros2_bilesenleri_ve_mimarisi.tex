% 02_ros2_bilesenleri_ve_mimarisi.tex
\chapter{ROS2 Bileşenleri ve Mimarisi}

\section{ROS2'nin Temel Bileşenleri ve İleri Düzey Kullanımı}
\subsection{Node'lar, Topic'ler, Service'ler ve Action'lar}
\subsubsection{İleri Düzey Node Kullanımı}
\lipsum[1] % Örnek metin için, gerçek içerikle değiştirin

\subsubsection{Etkin Topic ve Mesaj İşleme}
\lipsum[2] % Örnek metin için, gerçek içerikle değiştirin

\subsection{Kompozit ve Lifecycle Node'lar}
\subsubsection{Kompozit Node Kavramı ve Kullanımı}
\lipsum[3] % Örnek metin için, gerçek içerikle değiştirin

\subsubsection{Lifecycle Node'lar ve Yönetimi}
\lipsum[4] % Örnek metin için, gerçek içerikle değiştirin

\section{DDS (Data Distribution Service) ve QoS (Quality of Service) Parametreleri}
\subsection{DDS ve ROS2 İçin Önemi}
\lipsum[5] % Örnek metin için, gerçek içerikle değiştirin

\subsection{QoS Parametreleri ve Kullanımı}
\subsubsection{QoS Profilleri ve Ayarları}
\lipsum[6] % Örnek metin için, gerçek içerikle değiştirin

\subsubsection{QoS ile İletişim Kalitesini Artırma}
\lipsum[7] % Örnek metin için, gerçek içerikle değiştirin
