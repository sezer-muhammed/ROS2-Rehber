\chapter{ROS2 ile Simülasyon ve Donanım Entegrasyonu}

\section{Gazebo Simülasyonu ve ROS2 Entegrasyonu}
\subsection{Gazebo Simülasyonunun Tanıtılması}
Gazebo, bir 3D dinamik simülatör olan ve özellikle robotik uygulamalar için tasarlanmış bir yazılımdır. Gerçek dünyayı modellemek için fizik motorlarına sahiptir ve kullanıcıların robotları, sensörleri ve karmaşık ortamları 3 boyutlu olarak oluşturmasına izin verir.

\subsection{ROS2 İle Gazebo Simülasyon Entegrasyonu}
ROS2 ile Gazebo'nun entegrasyonu, simülasyon ortamında robotları test etme ve önceden belirlenmiş görevleri yerine getirme imkanı sağlar. Bu entegrasyon, ROS2'nin mesaj geçiş özellikleri ve Gazebo'nun dinamik simülasyon yeteneklerini birleştirerek karmaşık robotik sistemlerin geliştirilmesini kolaylaştırır.

\subsubsection{Gazebo Simülasyonu İçin ROS2 Paketleri}
Gazebo ile ROS2'yi entegre etmek için gazebo\_ros\_pkgs gibi bir dizi ROS2 paketi bulunmaktadır. Bu paketler, Gazebo'nun simülasyon yeteneklerini ROS2 ekosistemi ile kullanmak için gerekli araçları ve kütüphaneleri sağlar.

\subsubsection{Örnek Gazebo ve ROS2 Entegrasyon Senaryosu}
Gazebo ve ROS2 entegrasyonunun bir örneği olarak, bir robotun simülasyon ortamında belirli bir hedefe gitmesi ve orada bir görevi tamamlaması durumu düşünülebilir. Bu senaryo, robotun hareketini kontrol etmek ve sensör verilerini işlemek için ROS2 düğümlerini ve Gazebo simülasyon ortamını kullanır.

\section{Farklı Robot Donanımı ve Sensörlerle Çalışma}
\subsection{ROS2 ve Robot Donanım Entegrasyonu}
ROS2, farklı robot donanımlarıyla entegrasyonu kolaylaştırır. ROS2 düğümleri, sürücüler ve donanım arayüzleri kullanılarak bir dizi farklı robot donanımı kontrol edilebilir.

\subsection{ROS2 ve Sensör Entegrasyonu}
ROS2, farklı sensör türleriyle de entegre olabilir. Sensör verileri, ROS2 düğümleri ve mesajları kullanılarak toplanabilir ve işlenebilir, bu da robotik sistemlerin geniş bir sensör yelpazesinden veri almasını sağlar.

\subsubsection{Donanım ve Sensör Entegrasyonu İçin ROS2 Paketleri}
ROS2, farklı robot donanımları ve sensörlerle entegrasyonu kolaylaştıran çeşitli paketler sağlar. Bu paketler, sürücüler, donanım arayüzleri ve sensör verilerini işlemek için gereken araçları içerir.

\subsubsection{Örnek Donanım ve Sensör Entegrasyon Senaryosu}
Bir örnek senaryo olarak, bir ROS2 düğümü, bir lidar sensöründen gelen verileri toplayabilir ve bu verileri robotun çevresini algılamak ve haritalamak için kullanabilir. Bu tür bir entegrasyon, robotun çevresini daha iyi anlamasına ve daha etkili bir şekilde hareket etmesine olanak sağlar.